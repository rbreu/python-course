\section{Statements}

\begin{frame}[fragile]{Das if-Statement}
\begin{lstlisting}[style=Python]
if a == 3:
    print "Aha!"
\end{lstlisting}
\begin{itemize}
\item Bl"ocke werden durch Einr"uckung festgelegt!
\item Standard: Einr"uckung mit vier Leerzeichen
\end{itemize}
\begin{lstlisting}
if a == 3:
    print "spam"
elif a == 10:
    print "eggs"
elif a == -3:
    print "bacon"
else:
    print "something else"
\end{lstlisting}
\end{frame}

\begin{frame}[fragile]{Vergleichsoperatoren}
\begin{itemize}
\item Vergleich des Inhalts: \texttt{==}, \texttt{<}, \texttt{>}, \texttt{<=}, \texttt{>=}, \texttt{!=}
\item Vergleich der Objektidentit"at: \lstinline{a is b}, \lstinline{a is not b}\item Und/Oder-Verkn"upfung: \lstinline{a and b}, \lstinline{a or b}
\item Negation: \lstinline{not a}
\end{itemize}
\begin{lstlisting}
if not (a==b) and (c<3):
    pass
\end{lstlisting}
\end{frame}

\begin{frame}[fragile]{Conditional Expressions}
Kurze Schreibweise f"ur bedingte Zuweisung. Statt:
\begin{lstlisting}
if zahl<0:
    s = "Negativ"
else:
    s = "Positiv"
\end{lstlisting}
kann man schreiben:
\begin{lstlisting}
s = "Negativ" if zahl<0 else "Positiv"
\end{lstlisting}
\end{frame}

\begin{frame}[fragile]{for-Schleifen}
\begin{lstlisting}[style=Python]
for i in range(10):
    print i   # 0, 1, 2, 3, ..., 9

for i in range(3, 10):
   print i    # 3, 4, 5, ..., 9

for i in range(0, 10, 2):
   print i   # 0, 2, 4, ..., 8
else:
   print "Schleife komplett durchlaufen."
\end{lstlisting}
\begin{itemize}
\item Schleife vorzeitig beenden: \lstinline{break}
\item n"achster Durchlauf: \lstinline{continue}
\item \lstinline{else} wird ausgef"uhrt, wenn die Schleife nicht vorzeitig verlassen wurde
\end{itemize}
\end{frame}

\begin{frame}[fragile]
"Uber Sequenzen kann man direkt (ohne Index) iterieren:
\begin{lstlisting}[style=Python]
for item in ["spam", "eggs", "bacon"]:
    print item
\end{lstlisting}

Auch die \texttt{range}-Funktion liefert eine Liste:
\begin{lstlisting}[style=Shell]
>>> range(0, 10, 2)
[0, 2, 4, 6, 8]
\end{lstlisting}
Ben"otigt man doch Indices:
\begin{lstlisting}[style=Python]
for (i, char) in enumerate("hallo welt"):
    print i, char
\end{lstlisting}
\end{frame}

\begin{frame}[fragile]{while-Schleifen}
\begin{lstlisting}[style=Python]
while i < 10:
    i += 1
\end{lstlisting}
Auch hier k"onnen \lstinline{break} und \lstinline{continue} verwendet werden.\\[3mm]
Ersatz f"ur do-while-Schleife:
\begin{lstlisting}[style=Python]
while True:
   # wichtiger Code
   if bedingung:
       break
\end{lstlisting} 
\end{frame}

