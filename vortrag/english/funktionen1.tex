\section{Functions}

\begin{frame}[fragile]{Functions}
\begin{lstlisting}[style=Python]
def add(a, b):
    """Returns the sum of a and b."""

    mysum = a + b
    return mysum
\end{lstlisting}

\begin{lstlisting}[style=Shell]
>>> result = add(3, 5)
>>> print result
8
>>> help(add)
Help on function add in module __main__:

add(a, b)
    Returns the sum of a and b.
\end{lstlisting}
\end{frame}

\begin{frame}[fragile]{Return Values and Parameters}
\begin{itemize}
\item Functions accept arbitrary objects as parameters and return values
\item Types of parameters and return values are unspecified
\item Functions without explicit return value return \lstinline{None}
\end{itemize}
\begin{lstlisting}[style=Python]
def hello_world():
    print "Hello World!"

a = hello_world()
print a
\end{lstlisting}
\begin{lstlisting}[style=Shell]
$ my_program.py
Hello World
None
\end{lstlisting} %$
\end{frame}

\begin{frame}[fragile]{Multiple Return Values}
Multiple return values are realised using tuples or lists:
\begin{lstlisting}[style=Python]
def foo():
   a = 17
   b = 42
   return (a, b)

ret = foo()
(x, y) = foo()
\end{lstlisting}
\end{frame}

\begin{frame}[fragile]{Keywords and Default Values}
Parameters can be passed to a function in a different order than specified:
\begin{lstlisting}[style=Python]
def foo(a, b, c):
    print a, b, c

foo(b=3, c=1, a="hello")
\end{lstlisting}
Defining default values:
\begin{lstlisting}[style=Python]
def foo(a, b, c=1.3):
    print a, b, c

foo(1, 2)
foo(1, 17, 42)
\end{lstlisting}
\end{frame}

\begin{frame}[fragile]{Functions Are Objects}
Functions are objects and as such can be assigned and passed on:
\begin{lstlisting}[style=Shell]
>>> a = float
>>> a(22)
22.0
\end{lstlisting}
\begin{lstlisting}[style=Shell]
>>> def foo(fkt):
...     print fkt(33)
...
>>> foo(float)
33.0
\end{lstlisting}
\end{frame}

