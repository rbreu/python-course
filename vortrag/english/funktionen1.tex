\section{Funktionen}

\begin{frame}[fragile]{Funktionen}
\begin{lstlisting}[style=Python]
def add(a, b):
    """Returns the sum of a and b."""

    mysum = a + b
    return mysum
\end{lstlisting}

\begin{lstlisting}[style=Shell]
>>> result = add(3, 5)
>>> print result
8
>>> help(add)
Help on function add in module __main__:

add(a, b)
    Returns the sum of a and b.
\end{lstlisting}
\end{frame}

\begin{frame}[fragile]{R"uckgabewerte und Parameter}
\begin{itemize}
\item Funktionen k"onnen beliebige Objekte als Parameter und R"uckgabewerte haben
\item Typen der R"uckgabewerte und Parameter sind nicht festgelegt
\item Funktionen ohne expliziten R"uckgabewert geben \lstinline{None} zur"uck
\end{itemize}
\begin{lstlisting}[style=Python]
def hello_world():
    print "Hello World!"

a = hello_world()
print a
\end{lstlisting}
\begin{lstlisting}[style=Shell]
$ my_program.py
Hello World
None
\end{lstlisting} %$
\end{frame}

\begin{frame}[fragile]{Mehrere R"uckgabewerte}
Mehrere R"uckgabewerte werden mittels Tupel oder Listen realisiert:
\begin{lstlisting}[style=Python]
def foo():
   a = 17
   b = 42
   return (a, b)

ret = foo()
(x, y) = foo()
\end{lstlisting}
\end{frame}

\begin{frame}[fragile]{Keywords und Defaultwerte}
Man kann Parameter auch in anderer Reihenfolge als definiert angeben:
\begin{lstlisting}[style=Python]
def foo(a, b, c):
    print a, b, c

foo(b=3, c=1, a="hello")
\end{lstlisting}
Defaultwerte festlegen:
\begin{lstlisting}[style=Python]
def foo(a, b, c=1.3):
    print a, b, c

foo(1, 2)
foo(1, 17, 42)
\end{lstlisting}
\end{frame}

\begin{frame}[fragile]{Funktionen sind Objekte}
Funktionen sind Objekte und k"onnen wie solche zugewiesen und "ubergeben werden:
\begin{lstlisting}[style=Shell]
>>> a = float
>>> a(22)
22.0
\end{lstlisting}
\begin{lstlisting}[style=Shell]
>>> def foo(fkt):
...     print fkt(33)
...
>>> foo(float)
33.0
\end{lstlisting}
\end{frame}

