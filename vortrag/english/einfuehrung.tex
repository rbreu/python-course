\section{Introduction}

\begin{frame}{What is Python?}
\alert{Python:} dynamic programming language which supports several different programing paradigms:
\begin{itemize}
\item procedural programming
\item object oriented programming
\item functional programming
\end{itemize}
Standard: Python byte code is executed in the Python interpreter (similar to Java)\\
$\rightarrow$ \alert{platform independent code}
\end{frame}

\begin{frame}{Why Python?}
\begin{itemize}
\item syntax is clear, easy to read and learn (almost pseudo code)
\item intuitive object oriented programming
\item full modularity, hierarchical packages
\item error handling via exceptions
\item dynamic, high level data types
\item comprehensive standard library for many tasks
\item simply extendable via C/C++, wrapping of C/C++ libraries
\end{itemize}
\alert{Focus: programming speed}
\end{frame}

\begin{frame}{Is Python fast enough?}
\begin{itemize}
\item for compute intensive algorithms: Fortran, C, C++ might be better
\item for user programs: Python is fast enough!
\item most parts of Python are written in C
\item performance-critical parts can be re-implemented in C/C++ if necessary
\item first analyse, then optimise!
\end{itemize}
\end{frame}

\begin{frame}[fragile]{Hello World!}
\begin{lstlisting}[style=Python]
#!/usr/bin/env python

# This is a commentary
print "Hello world!"
\end{lstlisting}
\begin{lstlisting}[style=Shell]
$ python hello_world.py
Hello world!
$
\end{lstlisting}%$
\begin{lstlisting}[style=Shell]
$ chmod 755 hello_world.py
$ ./hello_world.py
Hello world!
$
\end{lstlisting} %$
\end{frame}

\begin{frame}[fragile]{Hello User}
\begin{lstlisting}[style=Python]
#!/usr/bin/env python

name = raw_input("What's your name? ")
print "Hello", name
\end{lstlisting}
\begin{lstlisting}[style=Shell]
$ ./hello_user.py
What's your name? Rebecca
Hello Rebecca
$
\end{lstlisting}
\end{frame}

\begin{frame}{Strong and Dynamic Typing}
\alert{Strong Typing:}
\begin{itemize}
\item Object is of exactly one type! A string is always a string, an integer always an integer
\item Counter examples: PHP, JavaScript, C: \texttt{char} can be interpreted as \texttt{short}, \texttt{void~*} can be everything
\end{itemize}
\alert{Dynamic Typing:}
\begin{itemize}
\item no variable declaration
\item variable names can be assigned to different data types in the course of a program
\item An object's attributes are checked only at run time
\end{itemize}
\end{frame}

\begin{frame}[fragile]{Strong and Dynamic Typing}
\begin{lstlisting}[style=Python]
number = 3
print number, type(number)
print number + 42
number = "3"
print number, type(number)
print number + 42
\end{lstlisting}
\begin{lstlisting}[style=Shell]
3 <type 'int'>
45
3 <type 'str'>
Traceback (most recent call last):
  File "test.py", line 6, in ?
    print number + 42
TypeError: cannot concatenate 'str' and 
'int' objects
\end{lstlisting}
\end{frame}

\begin{frame}[fragile]{Interactive Mode}
The interpreter can be started in interactive mode:
\begin{lstlisting}[style=Shell]
$ python
Python 2.6 (r26:66714, Feb  3 2009, 20:52:03) 
[GCC 4.3.2] on linux2
Type "help", "copyright", "credits" or ...
>>> print "hello world"
hello world
>>> a = 3 + 4
>>> print a
7
>>> 3 + 4
7
>>>
\end{lstlisting} %$
\end{frame}

\begin{frame}{Documentation}
Online help in the interpreter:
\begin{itemize}
\item \alert{\lstinline{help()}}: general Python help
\item \alert{\lstinline{help(obj)}}: help regarding an object, e.g. a function or a module
\item \alert{\lstinline{dir()}}: all used names
\item \alert{\lstinline{dir(obj)}}: all attributes of an object
\end{itemize}
\vspace{5mm}
Official documentation: \href{http://docs.python.org/}{http://docs.python.org/}
\end{frame}

\begin{frame}[fragile]{Documentation}
\begin{lstlisting}[style=Shell]
>>> help(dir)
Help on built-in function dir:
...
>>> a = 3
>>> dir()
['__builtins__', '__doc__', '__file__', 
'__name__', 'a']
>>> help(a)
Help on int object:
...
\end{lstlisting}
\end{frame}

