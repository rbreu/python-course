\section{Pythons Standardbibliothek}

\begin{frame}{Pythons Standardbibliothek}
\alert{\glqq Batteries included\grqq}: umfassende Standardbibliothek f"ur die verschiedensten Aufgaben\\[4mm]

\includegraphics[height=4.5cm]{images/batteries_included.jpg}
\end{frame}

\begin{frame}[fragile]{Mathematik: \texttt{math}}
\begin{itemize}
\item Konstanten: \texttt{e}, \texttt{pi}
\item Auf- und Abrunden: \texttt{floor(x)}, \texttt{ceil(x)}
\item Exponentialfunktion: \texttt{exp(x)}
\item Logarithmus: \texttt{log(x[, base])}, \texttt{log10(x)}
\item Potenz und Quadratwurzel: \texttt{pow(x, y)}, \texttt{sqrt(x)}
\item Trigonometrische Funktionen: \texttt{sin(x)}, \texttt{cos(x)}, \texttt{tan(x)}
\item Kovertierung Winkel $\leftrightarrow$ Radiant: \texttt{degrees(x)}, \texttt{radians(x)}
\end{itemize}
\begin{lstlisting}[style=Shell]
>>> import math
>>> math.sin(math.pi)
1.2246063538223773e-16
>>> math.cos(math.radians(30))
0.86602540378443871
\end{lstlisting}
\end{frame}


\begin{frame}[fragile]{Exakte Dezimalzahlen: fractions}

\begin{lstlisting}
>>> from fractions import Fraction
>>> a = Fraction(2, 3)
>>> b = Fraction(2, 5)
>>> print float(a), float(b)
0.66666666666666663 0.40000000000000002
>>> a+b
Fraction(16, 15)
>>> a/b
Fraction(5, 3)
\end{lstlisting}
\end{frame}


\begin{frame}[fragile]{Zufall: \texttt{random}}
\begin{itemize}
\item Zuf"allige Integers: \\ \texttt{randint(a, b)},  \texttt{randrange([start,] stop[, step])}
\item Zuf"allige Floats (Gleichverteilg.): \texttt{random()}, \texttt{uniform(a, b)}
\item Andere Verteilungen: \texttt{expovariate(lambd)}, \texttt{gammavariate(alpha, beta)}, \texttt{gauss(mu, sigma)}, \dots
\item Zuf"alliges Element einer Sequenz: \texttt{choice(seq)}
\item Mehrere eindeutige, zuf"allige Elemente einer Sequenz: \texttt{sample(population, k)}
\item Sequenz mischen: \texttt{shuffle(seq[, random])}
\end{itemize}
\begin{lstlisting}[style=Python]
>>> s = [1, 2, 3, 4, 5]
>>> random.shuffle(s)
>>> s
[2, 5, 4, 3, 1]
>>> random.choice("Hallo Welt!")
'e'
\end{lstlisting}
\end{frame}

\begin{frame}[fragile]{Operationen auf Verzeichnisnamen: \texttt{os.path}}
\begin{itemize}
\item Pfade: \texttt{abspath(path)}, \texttt{basename(path)}, \texttt{normpath(path)}, \texttt{realpath(path)}
\item Pfad zusammensetzen: \texttt{join(path1[, path2[, ...]])}
\item Pfade aufspalten: \texttt{split(path)}, \texttt{splitext(path)}
\item Datei-Informationen: \texttt{isfile(path)}, \texttt{isdir(path)}, \texttt{islink(path)}, \texttt{getsize(path)}, \dots
\item Home-Verzeichnis vervollst"andigen: \texttt{expanduser(path)}
\item Umgebungsvariablen vervollst"andigen: \texttt{expandvars(path)}
\end{itemize} 
\begin{lstlisting}[style=Shell]
>>> os.path.join("spam", "eggs", "ham.txt")
'spam/eggs/ham.txt'
>>> os.path.splitext("spam/eggs.py")
('spam/eggs', '.py')
>>> os.path.expanduser("~/spam")
'/home/rbreu/spam'
>>> os.path.expandvars("/bla/$TEST")
'/bla/test.py'
\end{lstlisting}%$
\end{frame}

\begin{frame}[fragile]{Dateien und Verzeichnisse: \texttt{os}}
\begin{itemize}
\item Working directory: \lstinline{getcwd()}, \lstinline{chdir(path)}
\item Dateirechte "andern: \lstinline{chmod(path, mode)}
\item Besitzer "andern: \lstinline{chown(path, uid, gid)}
\item Verzeichnis erstellen: \lstinline{mkdir(path[, mode])}, \lstinline{makedirs(path[, mode])}
\item Dateien l"oschen: \lstinline{remove(path)}, \lstinline{removedirs(path)}
\item Dateien umbenennen: \lstinline{rename(src, dst)}, \lstinline{renames(old, new)}
\item Liste von Dateien in Verzeichnis: \lstinline{listdir(path)}
\end{itemize}
\begin{lstlisting}[style=Python]
for datei in os.listdir("mydir"):
    os.chmod(os.path.join("mydir", datei), 
             stat.S_IRUSR|stat.S_IWUSR)
\end{lstlisting}
\end{frame} 

\begin{frame}[fragile]{Verzeichnislisting: \texttt{glob}}
Liste von Dateien in Verzeichnis, mit Unix-artiger Wildcard-Vervollst"andigung: \texttt{glob(path)}
\begin{lstlisting}[style=Shell]
>>> glob.glob("python/[a-c]*.py")
['python/confitest.py',
 'python/basics.py',
 'python/curses_test2.py',
 'python/curses_keys.py',
 'python/cmp.py',
 'python/button_test.py',
 'python/argument.py',
 'python/curses_test.py']
\end{lstlisting}
\end{frame}

\begin{frame}[fragile]{Dateien und Verzeichnisse: \texttt{shutil}}
Higher Level-Operationen auf Dateien und Verzeichnissen.
\begin{itemize}
\item Datei kopieren: \texttt{copyfile(src, dst)}, \texttt{copy(src, dst)}
\item Rekursiv kopieren; \texttt{copytree(src, dst[, symlinks])}
\item Rekursiv l"oschen: \\\texttt{rmtree(path[, ignore\_errors[, onerror]])}
\item Rekursiv verschieben: \texttt{move(src, dst)}
\end{itemize}
\begin{lstlisting}[style=Python]
shutil.copytree("spam/eggs", "../beans", 
                 symlinks=True)
\end{lstlisting}
\end{frame}

\begin{frame}[fragile]{Andere Prozesse starten: \texttt{subprocess}}
Einfaches Ausf"uhren eines Programmes:
\begin{lstlisting}[style=Python]
p = subprocess.Popen(["ls", "-l", "mydir"])
returncode = p.wait() # Auf Ende warten
\end{lstlisting}
Zugriff auf die Ausgabe eines Programmes:
\begin{lstlisting}[style=Python]
p = Popen(["ls"], stdout=PIPE, stderr=STDOUT) 
p.wait()
output = p.stdout.read()
\end{lstlisting}
Pipes zwischen Prozessen (\lstinline{ls -l | grep txt})
\begin{lstlisting}[style=Python]
p1 = Popen(["ls", "-l"], stdout=PIPE)
p2 = Popen(["grep", "txt"], stdin=p1.stdout)
\end{lstlisting}
\end{frame}

\begin{frame}[fragile]{Threads: \texttt{threading}}
Programmteile gleichzeitig ablaufen lassen mit \alert{Thread-Objekten}:
\begin{lstlisting}[style=Python]
class Counter(threading.Thread):
    def __init__(self):
        threading.Thread.__init__(self)
        self.counter = 0

    def run(self):  # Hauptteil
        while self.counter < 10:
            self.counter += 1
            print self.counter

counter = Counter()
counter.start() # Thread starten
# hier etwas gleichzeitig tun...
counter.join() # Warte auf Ende des Threads
\end{lstlisting}
\end{frame}

\begin{frame}[fragile]{Threads: \texttt{threading}}
\begin{itemize}
\item Problem, wenn zwei Threads gleichzeitig auf das gleiche Objekt schreibend zugreifen wollen!
\item $\rightarrow$ Verhindern, dass Programmteile gleichzeitig ausgef"uhrt werden mit \alert{Lock-Objekten}
\item Locks haben genau zwei Zust"ande: \texttt{locked} und \texttt{unlocked}
\end{itemize}
\begin{lstlisting}[style=Python]
lock = threading.Lock()

lock.aquire() # Warte bis Lock frei ist
              # und locke es dann 
#... wichtiger Code
lock.release() # Lock freigeben fuer andere
\end{lstlisting}
\end{frame}

\begin{frame}{Threads: \texttt{threading}}
\begin{itemize}
\item Kommunikation zwischen Threads: Z.B. mittels \alert{Event-Objekten}
\item Events haben zwei Zust"ande: gesetzt und nicht gesetzt
\item "ahnlich Locks, aber ohne gegenseitigen Ausschluss
\end{itemize}
Bsp: Event, um Threads mitzuteilen, dass sie sich beenden sollen.\\
Methoden auf Event-Objekten:
\begin{itemize}
\item Status des Events abfragen: \texttt{isSet()}
\item Setzen des Events: \texttt{set()}
\item Zur"ucksetzten des Events: \texttt{clear()}
\item Warten, dass Event gesetzt wird: \texttt{wait([timeout])}
\end{itemize}
\end{frame}

\begin{frame}[fragile]{Zugriff auf Kommandozeilenparameter: \texttt{optparse}}
\begin{itemize}
\item Einfach: Liste mit Parametern $\rightarrow$ \texttt{sys.argv}
\item Komfortabler f"ur mehrere Optionen: \texttt{OptionParser}
\end{itemize}
\begin{lstlisting}[style=Python]
parser = optparse.OptionParser()
parser.add_option("-f", "--file", 
                  dest="filename",
                  default="out.txt",
                  help="output file")
parser.add_option("-v", "--verbose",
                  action="store_true", 
                  dest="verbose", 
                  default=False,
                  help="verbose output")

(options, args) = parser.parse_args()
print options.filename, options.verbose
print args
\end{lstlisting}
\end{frame}
\begin{frame}[fragile]{Zugriff auf Kommandozeilenparameter: \texttt{optparse}}
So wird ein optparse-Programm verwendet:
\begin{lstlisting}[style=Shell]
$ ./test.py -h
usage: test.py [options]

options:
  -h, --help     show this help message and exit
  -f FILENAME, --file=FILENAME
                 output file
  -v, --verbose  verbose output
\end{lstlisting} %$
\begin{lstlisting}[style=Shell]
$ ./test.py -f aa bb cc
aa False
['bb', 'cc']
\end{lstlisting} %$
\end{frame}

\begin{frame}[fragile]{Konfigurationsdateien: \texttt{ConfigParser}}
Einfaches Format zum Speichern von Konfigurationen u."A.: Windows INI-Format
\begin{lstlisting}[style=Python]
[font]
font = Times New Roman
# Kommentar (oder: ! als Kommentarzeichen)
size = 16

[colors]
font = black
pointer = %(font)s
background = white
\end{lstlisting}
\end{frame}

\begin{frame}[fragile]{Konfigurationsdateien: \texttt{ConfigParser}}
Config-Datei lesen:
\begin{lstlisting}[style=Python]
parser = ConfigParser.SafeConfigParser()
parser.readfp(open("config.ini", "r"))
print parser.get("colors", "font")
\end{lstlisting}
Weitere Parser-Methoden:
\begin{itemize}
\item Liste aller Sections: \texttt{sections()}
\item Liste aller Optionen: \texttt{options(section)}
\item Liste aller Optionen und Werte: \texttt{items(section)}
\item Werte lesen: \texttt{get(sect, opt)}, \\
\texttt{getint(sect, opt)}, \texttt{getfloat(sect, opt)}, \texttt{getboolean(sect, opt)}
\end{itemize}
\end{frame}

\begin{frame}[fragile]{Konfigurationsdateien: \texttt{ConfigParser}}
Config-Datei schreiben:
\begin{lstlisting}[style=Python]
parser = ConfigParser.SafeConfigParser()
parser.add_section("colors")
parser.set("colors", "font", "black")
parser.write(open("config.ini", "w"))
\end{lstlisting}
Weitere Parser-Methoden:
\begin{itemize}
\item Section hinzuf"ugen: \texttt{add\_section(section)}
\item Section l"oschen: \texttt{remove\_section(section)}
\item Option hinzuf"ugen: \texttt{set(section, option, value)}
\item Option entfernen: \texttt{remove\_option(section, option)}
\end{itemize}
\end{frame}

\begin{frame}[fragile]{CSV-Dateien: \texttt{csv}}
CSV: Comma-seperated values
\begin{itemize}
\item Tabellendaten im ASCII-Format
\item Spalten durch ein festgelegtes Zeichen (meist Komma) getrennt
\end{itemize}
\begin{lstlisting}[style=Python]
reader = csv.reader(open("test.csv", "rb"))
for row in reader:
    for item in row: 
        print item
\end{lstlisting}
\begin{lstlisting}[style=Python]
writer = csv.writer(open(outfile, "wb"))
writer.writerow([1, 2, 3, 4])
\end{lstlisting}
\end{frame}

\begin{frame}[fragile]{CSV-Dateien: \texttt{csv}}
Mit verschiedenen Formaten (Dialekten) umgehen:
\begin{lstlisting}[style=Python]
reader(csvfile, dialect='excel') # Default
writer(csvfile, dialect='excel_tab')
\end{lstlisting}
\vspace*{3mm}
Einzelne Formatparameter angeben:
\begin{lstlisting}[style=Python]
reader(csvfile, delimiter=";")
\end{lstlisting}
Weitere Formatparameter: \texttt{lineterminator}, \texttt{quotechar}, \texttt{skipinitialspace}, \dots
\end{frame}

\begin{frame}[fragile]{Objekte serialisieren: \texttt{pickle}}
Beliebige Objekte in Dateien speichern:
\begin{lstlisting}[style=Python]
obj = {"hallo": "welt", "spam":1}
pickle.dump(obj, open("bla.bin", "wb"))
# ...
obj = pickle.load(open("bla.bin", "rb"))
\end{lstlisting}
Objekt in String unwandeln (z.B. zum Verschicken "uber Streams):
\begin{lstlisting}[style=Python]
s = pickle.dumps(obj)
# ...
obj = pickle.loads(s)
\end{lstlisting}
\end{frame}

\begin{frame}[fragile]{Persistente Dictionaries: \texttt{shelve}}
Ein Shelve benutzt man wie ein Dictionary, es speichert seinen Inhalt in eine Datei.
\begin{lstlisting}[style=Python]
d = shelve.open("bla")
d["spam"] = "eggs"
d["bla"] = 1
del d["foo"]   
d.close()  
\end{lstlisting}
\end{frame}

\begin{frame}[fragile]{Leichtgewichtige Datenbank: \texttt{sqlite3}}
Datenbank in Datei oder im Memory, ab Python 2.5 in der stdlib.
\begin{lstlisting}[style=Python]
conn = sqlite3.connect("bla.db")
c = conn.cursor()

c.execute("""CREATE TABLE Friends
             (vorname TEXT, nachname TEXT)""")
c.execute("""INSERT INTO Friends
             VALUES("Max", "Mueller")""")
conn.commit()
\end{lstlisting}
\begin{lstlisting}[style=Python]
c.execute("""SELECT * FROM Friends""")
for row in c: print row

c.close(); conn.close()
\end{lstlisting}
\end{frame}

\begin{frame}[fragile]{Leichtgewichtige Datenbank: \texttt{sqlite3}}
String-Formatter sind unsicher, da beliebiger SQL-Code eingeschleust werden kann!
\begin{lstlisting}[style=Python]
# Never do this!
symbol = "Max"
c.execute("... WHERE name = '%s'" % symbol)
\end{lstlisting}
Stattdessen die Platzhalter der Datenbank-API benutzen:
\begin{lstlisting}[style=Python]
c.execute("... WHERE name = ?", symbol)
\end{lstlisting}
\begin{lstlisting}[style=Python]
f = (("Alice", "Meier"), ("Bob", "Schmidt"))
for item in friends:
    c.execute("""INSERT INTO Friends 
                 VALUES (?,?)""", item)
\end{lstlisting}
\end{frame}

\begin{frame}[fragile]{Tar-Archive: \texttt{tarfile}}
Ein tgz entpacken:
\begin{lstlisting}[style=Python]
tar = tarfile.open("spam.tgz")
tar.extractall()
tar.close()
\end{lstlisting}
\vspace*{3mm}
Ein tgz erstellen:
\begin{lstlisting}[style=Python]
tar = tarfile.open("spam.tgz", "w:gz")
tar.add("/home/rbreu/test")
tar.close()
\end{lstlisting}
\end{frame}

\begin{frame}[fragile]{Log-Ausgaben: \texttt{logging}}
Flexible Ausgabe von Informationen, kann schnell angepasst werden.
\begin{lstlisting}[style=Python]
import logging
logging.debug("Very special information.")
logging.info("I am doing this and that.")
logging.warning("You should know this.")
\end{lstlisting}
\begin{lstlisting}[style=shell]
WARNING:root:You should know this.
\end{lstlisting}
\begin{itemize}
\item Messages bekommen einen Rang (Dringlichkeit):\\
 \texttt{CRITICAL}, \texttt{ERROR}, \texttt{WARNING}, \texttt{INFO}, \texttt{DEBUG}
\item Default: Nur Messages mit Rang \texttt{WARNING} oder h"oher werden ausgegeben
\end{itemize}
\end{frame}

\begin{frame}[fragile]{Log-Ausgaben: \texttt{logging}}
Beispiel: Ausgabe in Datei, benutzerdefiniertes Format, feineres Log-Level:
\begin{lstlisting}[style=Python, basicstyle=\ttfamily\small]
logging.basicConfig(level=logging.DEBUG,
  format="%(asctime)s %(levelname)-8s %(message)s",
  datefmt="%Y-%m-%d %H:%M:%S",
  filename='/tmp/spam.log', filemode='w')
\end{lstlisting}
\begin{lstlisting}[style=Shell, basicstyle=\ttfamily\small]
$ cat /tmp/spam.log
2007-05-07 16:25:14 DEBUG   Very special information.
2007-05-07 16:25:14 INFO    I am doing this and that.
2007-05-07 16:25:14 WARNING You should know this.
\end{lstlisting} %$
Es k"onnen auch verschiedene Loginstanzen gleichzeitig benutzt werden,
siehe Python-Dokumentation.
\end{frame}

\begin{frame}[fragile]{Regul"are Ausdr"ucke: \texttt{re}}
Einfaches Suchen nach Mustern:
\begin{lstlisting}[style=Shell]
>>> re.findall(r"\[.*?\]", "a[bc]g[hal]def")
['[bc]', '[hal]']
\end{lstlisting}
\vspace*{2mm}
Ersetzen von Mustern:
\begin{lstlisting}[style=Shell]
>>> re.sub(r"\[.*?\]", "!", "a[bc]g[hal]def")
'a!g!def'
\end{lstlisting}
\vspace*{2mm}
Wird ein Regex-Muster mehrfach verwendet, sollte es aus Geschwindigkeitsgr"unden compiliert werden:
\begin{lstlisting}[style=Shell]
>>> pattern = re.compile(r"\[.*?\]")
>>> pattern.findall("a[bc]g[hal]def")
['[bc]', '[hal]']
\end{lstlisting}
\end{frame}

\begin{frame}[fragile]{Regul"are Ausdr"ucke: \texttt{re}}
Umgang mit Gruppen:
\begin{lstlisting}[style=Shell]
>>> re.findall("(\[.*?\])|(<.*?>)", 
               "[hi]s<b>sdd<hal>")
[('[hi]', ''), ('', '<b>'), ('', '<hal>')]
\end{lstlisting}
\vspace*{2mm}
Flags, die das Verhalten des Matching beeinflussen:
\begin{lstlisting}[style=Shell]
>>> re.findall("^a", "abc\nAbc", re.I|re.M)
>>> ['a', 'A']
\end{lstlisting}
\begin{itemize}
\item \texttt{re.I}: Gro"s-/Kleinschreibung ingnorieren
\item \texttt{re.M}: \lstinline{^} bzw. \lstinline{$} matchen am Anfang/Ende jeder Zeile (nicht nur am Anfang des Strings) %$
\item \texttt{re.S}: \lstinline{.} matcht auch Zeilenumbruch
\end{itemize}
\end{frame}

\begin{frame}[fragile]{Sockets: \texttt{socket}}
Client-Socket erstellen und mit Server verbinden:
\begin{lstlisting}[style=Python]
sock = socket.socket(socket.AF_INET, 
                     socket.SOCK_STREAM)
sock.connect(("whois.denic.de", 43))
\end{lstlisting}
\vspace*{2mm}
Mit dem Server kommunizieren:
\begin{lstlisting}[style=Python]
sock.send("-T dn fz-juelich.de\n")
print sock.recv(4096) # Antwort lesen
sock.close()
\end{lstlisting}
\end{frame}

\begin{frame}[fragile]{Sockets: \texttt{socket}}
Server-Socket erstellen:
\begin{lstlisting}[style=Python]
server_socket = socket.socket(socket.AF_INET)
server_socket.bind(("localhost", 6666))
\end{lstlisting}
\vspace*{2mm}
Auf Client-Verbindungen warten und sie akzeptieren:
\begin{lstlisting}[style=Python]
server_socket.listen(1)
(sock, address) = server_socket.accept()
\end{lstlisting}
\vspace*{2mm}
Mit dem Client kommunizieren:
\begin{lstlisting}[style=Python]
sock.send("Willkommen!\n")
# ...
\end{lstlisting}
\end{frame}

\begin{frame}[fragile]{XML-RPC-Client: \texttt{xmlrpclib}}
\begin{itemize}
\item XML-RPC: \alert{Remote Procedure Call} via XML und HTTP
\item unabh"anging von Plattform und Programmiersprache
\end{itemize}
\begin{lstlisting}[style=Python]
import xmlrpclib

s = xmlrpclib.Server("http://localhost:8000")
print s.add(2,3) 
print s.sub(5,2) 
\end{lstlisting}
Konvertierungen f"ur die g"angigen Datentypen geschehen automatisch: 
Booleans, Integer, Floats, Strings, Tupel, Listen, Dictionaries mit Strings als Keys, \dots
\end{frame}

\begin{frame}[fragile]{XML-RPC-Server: \texttt{SimpleXMLRPCServer}}
\begin{lstlisting}[style=Python]
from SimpleXMLRPCServer import SimpleXMLRPCServer

# Methoden, die der Server zur Verfuegung
# stellen soll:
class MyFuncs:
    def add(self, x, y):
        return x + y
    def sub(self, x, y):
        return x - y
    
# Erstelle und starte Server:
server = SimpleXMLRPCServer(("localhost", 8000))
server.register_instance(MyFuncs())
server.serve_forever()
\end{lstlisting}
\end{frame}


