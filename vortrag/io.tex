\section{Input/Output}

\begin{frame}[fragile]{String-Formatierung}
Stringformatierung "ahnlich C: 
\begin{lstlisting}[style=Python]
print "The answer is %i." % 42
s = "%s: %3.4f" % ("spam", 3.14)
\end{lstlisting}
\begin{itemize}
\item \alert{Integer dezimal}: d, i
\item Integer oktal: o
\item Integer hexadezimal: x, X
\item \alert{Float}: f, F
\item Float in Exponentialdarstellung: e, E, g, G
\item Einzelnes Zeichen: c
\item \alert{String}: s
\end{itemize}
Ein \texttt{\%}-Zeichen gibt man als \texttt{\%\%} aus.
\end{frame}

\begin{frame}[fragile]{Kommandozeilen-Eingaben}
Benutzer-Eingaben:
\begin{lstlisting}[style=Python]
user_input = raw_input("Type something: ")
\end{lstlisting}
\vspace{3mm}
Kommandozeilen-Parameter:
\begin{lstlisting}[style=Python]
import sys
print sys.argv
\end{lstlisting}
\begin{lstlisting}[style=Shell]
$ ./params.py spam
['params.py', 'spam']
\end{lstlisting} %$
\end{frame}

\begin{frame}[fragile]{Dateien}
\begin{lstlisting}[style=Python]
file1 = open("spam", "r")
file2 = open("/tmp/eggs", "wb")
\end{lstlisting}
\begin{itemize}
\item Lesemodus: r
\item Schreibmodus: w
\item Bin"ardateien behandeln: b
\item Schreibmodus, an Daten am Ende anh"angen: a
\item Lesen und schreiben: r+
\end{itemize}
\begin{lstlisting}
for line in file1:
    print line
\end{lstlisting}
\end{frame}

\begin{frame}[fragile]{Operationen auf Dateien}
\begin{itemize}
\item \alert{lesen}: \lstinline{f.read([size])}
\item Zeile lesen: \lstinline{f.readline()}
\item mehrere Zeilen lesen: \lstinline{f.readlines([sizehint])}
\item \alert{schreiben}: \lstinline{f.write(str)}
\item mehrere Zeilen schreiben: \lstinline{f.writelines(sequence)}
\item Datei \alert{schlie"sen}: \lstinline{f.close()}
\end{itemize}
\begin{lstlisting}[style=Python]
file1 = open("test", "w")
lines = ["spam\n", "eggs\n", "ham\n"]
file1.writelines(lines)
file1.close()
\end{lstlisting}
Python wandelt \lstinline{\n} automatisch in den richtigen Zeilenumruch um!
\end{frame}
