\section*{Zus"atzliche Aufgaben}

\begin{aufgabe}[Zahlenraten]
Programmieren Sie das Spiel Zahlenraten: Ein Spieler (hier der Computer) denkt sich eine Zahl zwischen 1 und 100 aus, der andere Spieler (hier der Benutzer) muss die Zahl raten. Nach jedem Versuch erf"ahrt der Benutzer, ob er zu hoch, zu niedrig oder genau richtig lag. Eine Runde ist zuende, wenn die Zahl richtig geraden wurde.

Realisieren Sie das Programm in folgenden Schritten:
\begin{teilaufgabe}
Das Programm l"asst den Benutzer eine Runde spielen und gibt am Ende die Anzahl der ben"otigten Versuche aus.

\hinweis Eine Zufallszahl zwischen 1 und 100 wird mit dem Befehl \lstinline{random.randint(1, 100)} erzeugt.
\end{teilaufgabe}
\begin{teilaufgabe}
Das Programm fragt nach jeder Runde, ob der Benutzer noch einmal spielen m"ochte.
\end{teilaufgabe}
\begin{teilaufgabe}
Das Programm speichert eine Highscore-Liste in einer Datei.
\end{teilaufgabe}

\end{aufgabe}


