\section*{Input/Output}
\begin{aufgabe}[String-Formatierung]
Starten Sie den Python-Interpreter. 
\begin{auflistung}
\item Geben Sie \lstinline{float}-Zahlen mit festgelegter Anzahl an Vor- und Nachkommastellen aus und in Exponentialdarstellung.
\item Geben Sie \lstinline{int}-Zahlen in Hexadezimal- und Oktaldarstellung aus.
\item Sie wissen "uber eine Person die Daten \lstinline{vorname}, \lstinline{nachname}, \lstinline{wohnort}, \lstinline{alter}. Geben Sie die Daten so aus, dass Sie z.B. folgende Zeile erhalten: 
\begin{verbatim}
"James Kirk ist 35 Jahre alt und wohnt in Iowa."
\end{verbatim}
\item Probieren Sie weitere Formatierungen aus:\\ \texttt{\underline{http://docs.python.org/lib/typesseq-strings.html}}
\end{auflistung}
\end{aufgabe}

\begin{aufgabe}[Kommandozeilenparameter]
Schreiben Sie ein Programm, welches seinen Namen und alle ihm "ubergebenen Parameter ausgibt.
\end{aufgabe}

\begin{aufgabe}[Dateien lesen]
Schreiben Sie ein Programm, welches die H"aufigkeit des Wortes \glqq Spam\grqq{} in einer Datei z"ahlt. \hinweis{Es gibt eine hilfreiche String-Methode!}
\end{aufgabe}

\begin{aufgabe}[Dateien lesen und schreiben]
Schreiben Sie ein Programm, welches eine Datei einliest, jeder Zeile eine Zeilennummer hinzuf"ugt und dieses in einer neuen Datei speichert. Beispiel-Ausgabe:
\begin{verbatim}
1. Hallo
2. Welt
\end{verbatim}

\end{aufgabe}


